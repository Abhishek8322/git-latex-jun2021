\hrule

\section{MM20B020 Gokul C}

\subsection{My favourite equation - Fourier series}

\begin{equation}
f(t)=\sum_{n=-\infty}^{\infty} c_n e^{2 \pi n i t}
\label{eqn:mm20b020_1}
\end{equation}

\begin{equation}
c_n=\int  e^{-2 \pi n i t} f(t)\ dt
\label{eqn:mm20b020_2}
\end{equation}

\subsubsection{Description}
\paragraph{Fourier series} (Equation-\ref{eqn:mm20b020_1}) is a periodic function formed by weighted summation of harmincally related sinusoids. \cite{MITOPENCOURSEWARE_mm20b020}

Any function can be written as fourier series by using appropriate weights. Theweights can be calculated by using Equation-\ref{eqn:mm20b020_2}

Fourier series is an infinite series, but in practically it is impossible to compute a function of infinite length. Generally a finite number of terms(N) are considered, and the resolution of fourier series increases with increasing N, Figure-1 illustrates the above fact.

The computation of fourier series can be done by computers with the help sophisticated alogorithms. \cite{Cooley_mm20b020}

\begin{figure}[H]
	\begin{center}
		\includegraphics[width=200pt, height=200pt]{Fourier_series.eps}
	\end{center}
	\caption{mm20b020 Resolution of Fourier series with N\cite{Wikipedia_mm20b020}} 
	\label{fig:mm20b020}
\end{figure}

\subsubsection{Importance}
\begin{enumerate}
	\item{Signal Processing}
	\item{JPEG compression}
	\item{Solve partial differential equation}
	\item{To seperate sounds of different frequency from microphone input}
\end{enumerate}
.... and the list goes on

\hrule
